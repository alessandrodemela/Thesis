\chapter{Conclusions}
\lettrine{A}{} search for new phenomena in events with a high \ET photon and \met was performed. The interpretation of the results in terms of the Minimal DM model parameters has been presented in this thesis.
The analysis, made on 2015-2016 data collected with the ATLAS detector at a \cm energy of \SI{13}{\tev} corresponding to an integrated luminosity of \SI{36.4}{\ifb}, reported no excess of events in the SR and a good agreement between the results and the SM prediction was found. 
This agreement was translated into exclusion limits on the presence of physics BSM. A model independent upper limit on the visible cross section $\sigma\times A\times\epsilon$ of new phenomena, of \SI{5.91}{fb} at \SI{95}{\percent} CL was set.

This information was then used to set exclusion limits on \chizero masses in the context of the Minimal DM model. Several MC samples have been generated, using the ATLAS production chain, each one characterized by a different value of \chizero mass. These signal samples were analyzed in a ``fiducial region'', sharing the same kinematic cuts of the SR but at particle level, and the cross section upper limit was rescaled by taking into account the relative fiducial acceptance and fiducial reconstruction efficiency. Then a comparison between the theoretical cross section and the upper limit cross section was made for each sample. The \mph analysis excluded masses below \SI{50}{\gev} at \SI{95}{\percent} CL.

Finally, a projection of the result at higher integrated luminosities was performed. To simulate the full \RunTwo data taking, the integrated luminosity of \SI{120}{\ifb} was considered and masses below \SI{75}{\gev} are expected to be excluded at \SI{95}{\percent} CL. Moreover, to make prediction on the whole LHC activity program, an integrated luminosity of \SI{3000}{\ifb} was considered. The \mph analysis would exclude masses below \SI{350}{\gev} at \SI{95}{\percent} CL.



