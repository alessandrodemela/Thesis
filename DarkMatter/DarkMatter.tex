\chapter{Dark Matter}
One of the most challenging question of current physics is the nature of Dark Matter. Several evidences discussed in this chapter proves that in the Universe there is non-luminous matter and most of it is of non-baryonic nature. Currently data imply that ordinary matter accounts for about \num{4.56\pm0.16}\% while Dark Matter accounts for \num{22.7\pm1.4}\% and Dark Energy for the remaining \num{72.8\pm1.5}\% \cite{komatsu:abundance}.

Dark matter would constitute the principal nonbaryonic contribution to the matter density of the universe. Today DM particles can be produced at collider such LHC if it interacts with Standard Model (SM) particle via couplings at the electroweak scale, On the other hand they certainly cannot be detected but their interaction with the detector is seen as missing energy (\met) defined \textbf{later}. 

\section{The Standard Model of particle physics}
The Standard Model (SM) of particle physics is well established theory of elementary particles and their interaction. Up to now it is considered the most satisfying theory including three of the four fundamental forces.
Particles of SM include Spin $1/2$ Fermions (quarks and leptons), four force carrier particle i.e. the Spin 1 Gauge Bosons, which are the photon the \Zboson and \Wboson bosons and 8 gluons, and the Spin 0 Higgs Boson.
\\
 

\lipsum