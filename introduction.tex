\chapter{Introduction}
\lettrine{A}{fter} the discovery of the Higgs boson, many experiment at LHC focused on discovering physics beyond Standard Model. Despite the SM of elementary p articles is a common accepted and well established theory it has many open issue s such the gauge hierarchy problem and {\bfseries other problem introduced in Feng's paper}.
  
  Leading theories proposing to solve this open issues having the common feature of searching for BSM physics are SUSY, large extra dimensions theories and dark matter in the form of weakly interacting massive particles (WIMPs). In other words many attempts are being carried out by theoretical phisicist to find a satisfactory extension of the SM.

  Evidence of new physics might come from missing transverse energy (\MET) in scattering processes. One of the experiment performed by JDM group with the ATLAS detector at LHC is the Mono-Photon Analysis. \MET can of course be trivially associated with neutrinos production but there are other possible explanation to this lack of energy. For instance there are many theories stating that the so-called Dark Matter could be produced in proton proton scattering at the \TeV scale. In this work a new model proposed by M.Cirelli et al. is analyzed. It extends the SM with a EW fermion triplet, which is stable thanks to one of the symmetries already presented in the theory, composed by two charged and one un-charged particles which we assume to be the candidate for DM.

  Since it is a very simple model it is called Minimal Dark Matter model. Unlike other models it has only one parameter from the whole dynamics depends which is the mass of the DM particle.
  
  S  

  Analysing the data collected in 2015 and 2016 with an integrated luminosity of 36.4 \ifb my purpose was to set upper limits on fiducial, but maybe even on the visible, on WIMP production after giving a model independent upper limit on the same physical quantity too. I extimated the main background events that could condition the experiment and reported the outcomes. These events are provided by Monte-Carlo simulation or data-driven techniques normalizing, or fitting, the extimation on data. I took into account both experimental and systematic errors whose contribution is also reported. The techique involved is called ``Trasfer Factor'' techique which provides a parameter $k$ for every Control Region defined which acts as a normalization constant of MC events. 

  No excess of signal are found in the Signal Region and the results are in accordance with the expectation of the SM. 
  
  Dark Matter candidates are \dots

  Results are interpreted in terms of exclusion limits on fiducial and maybe visible cross section of new phoenomena. Here, you can present te actual results.
  
  Only for illustrative purposes, I'm going to give an overview of what could be found in this thesis. In chapter 1 MDM model is introduced, along with physical motivation of looking for DM. Chapter 2 introduces the LHC facility at CERN and the ATLAS experiment describing the detector and how it works. Fundamentals of MonoPhoton analysis are given in Chapter 4 correlated with a theorical statistical framework. Chapter 5 gets into the heart of the analysis. Here I will report the results for MDM model %4 and 5 potrebbero andare insieme, 5 magari outlook. 1 e 2 potrebbero invertirsi.
  %Darkmatter spiegando la ricerca ai collider e la mono-X+ LHC + MDM con generazione dei campioni fino a validation plots + analysis con bkg fino al limite m.i .
