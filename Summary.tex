\documentclass[a4paper,11pt]{article}
\usepackage[a-1b]{pdfx}
\newcommand{\ATLASLATEXPATH}{../atlaslatex/latex/}
\usepackage{\ATLASLATEXPATH atlasphysics}
\usepackage[range-phrase=-]{siunitx}
\usepackage{microtype}
\newcommand{\znng}{\mbox{$\Zboson (\rarrow \nu\nu )  + \gamma$} }
\newcommand{\zg}{\mbox{$\Zboson (\rarrow \ell\ell) + \gamma$} }
\newcommand{\wg}{\mbox{$\Wboson (\rarrow \lnu) + \gamma$} }
\newcommand{\gj}{\mbox{$\gamma + \text{jets}$} }
\newcommand{\mph}{MonoPhoton }
\newcommand{\cm}{centre-of-mass }
\newcommand{\chizero}{$\chi_0$ }
\newcommand{\chip}{$\chi^+$ }
\newcommand{\chim}{$\chi^-$ }
\newcommand{\chipm}{$\chi^\pm$ }


\usepackage[T1]{fontenc}
\usepackage[urw-garamond]{mathdesign}

\begin{document}

After the discovery of the Higgs boson, the experiments at the Large Hadron Collider (LHC) focused on the search for physics beyond the Standard Model (BSM). Despite the Standard Model (SM) of elementary particles is a common accepted and well established theory, it has many open issues such as the absence of a Dark Matter (DM) candidate. There are in fact strong cosmological and astrophysical observations indicating that the mass-energy content of the universe can't be accounted for by the known consituents in the SM.
  
One possible Dark Matter (DM) candidate would be a Weakly Interactive Massive Particle (WIMP). The WIMP idea is an interesting proposal thanks to the so called ``WIMP miracle'' stating that a particle with weak coupling to SM and with a mass in the \SI{1}{\gev}-\SI{1}{\tev} range can reproduce very well the relic density of DM measured by various experiments such the PLANCK mission by ESA. A huge effort is accomplished today to collect evidences for Dark Matter. The search comprehends direct detection experiments, in which the recoil of SM particles from a scattering with a WIMP is measured, indirect detection, in which SM annihilation products of DM are looked for, and colliders production in which DM comes from SM particles annihilation.

The \mph analysis is one of the searches for DM performed with the ATLAS detector at the LHC. In this analysis we look for events with a single energetic photon and large missing transverse momentum (\met) in \pp collisions. \met signatures are striking clues of eventual new phenomena. On the other hand the production of a couple of non interacting particles alone cannot be detected, instead it must be tagged with a well defined physics object such a photon or a jet. These kind of analyses are called \emph{Mono-X}. The MonoJet exhibits the higher production rate, while the \mph has the cleanest final state. The analysis performed in this thesis is pursued following the lead of the one published by the ATLAS Collaboration in 2017.

Data collected in 2015 and 2016 corresponding to an integrated luminosity of \SI{36.4}{\ifb} at a \cm energy of \SI{13}{\tev} are used. The main background source for the \mph analysis is the $\Zboson\gamma$ process, where a $\gamma$ is produced by initial state radiation and the \Zboson decays into 2 neutrinos. This is an irreducible background because it presents the same final state as the signal. Additional sources of backgrounds come from $\Zboson\gamma$ and $\Wboson\gamma$ processes with a leptonic decay of the \Zboson/\Wboson boson when one or two leptons are not reconstructed in the detector. Morover events from the $\Zboson\left(\nu\nu\right)+\text{jet}$ or the $\Wboson\left(\ell\nu\right)+\text{jet}$ processes in which a photon could be faked by an electron or a jet are taken into account. The analysis strategy involves the definition of a Signal Region (SR), which maximizes the probability of observing a signal, and several Control Regions (CR) used to estimate the background sources in the SR by a simultaneous fitting technique which provides a normalization factor $k$ for every MC background process. No excess of events is found in the SR and the results are in agreement with the expectations of the SM.

In this thesis a new interpretation of the results of the \mph analysis is provided in terms of the Mimimal Dark Matter model. This model simply adds to the SM a fermionic triplet, made of two opposite charged particles, $\chi^\pm$, and a neutral one, $\chi_0$, which can be considered an eventual WIMP. \chizero mass is the only free parameter of this model and in this work the mass range \SIrange{1}{1200}{\gev} \mbox{is studied}.

  
For this study a MonteCarlo production of several signal samples, \mbox{corresponding} to different \chizero masses, has been performed. The whole ATLAS production chain has been followed: it comprehends the generation of the events, the detector simulation, the digitization of the events and the physiscs object reconstruction. The output format of this process contains only information and events useful for the \mph analysis. 

The statistical analysis is based on a profile likelihood function which includes information from the various regions and a profile likelihood ratio test statistic is used in order to evaluate the significance of an excess or to set limits on the production cross section.

Results of the \mph analysis are interpreted in terms of model independent limit on the visible cross section of new phenomena ($\sigma\times A\times\epsilon$) which, at \SI{95}{\percent} CL, is \SI{5.91}{fb} (\SI{8.86}{fb} expected). A fiducial region is defined to permit the reinterpretation in terms of the Minimal Dark Matter model. The fiducial acceptance and the fiducial reconstruction efficiency are computed and limits on excluded masses for the Minimal DM model are extracted. This analysis sets an upper limit on the \chizero mass of \SI{50}{\gev}.

Projections at higher integrated luminosity are also performed: the computed upper limit for \chizero mass at \SI{120}{\ifb}, which corresponds to the full \RunTwo luminosity is \SI{75}{\gev}, while considering the total high luminosity phase of LHC, \mbox{corresponding} to \SI{3000}{\ifb}, the upper limit on \chizero mass is shifted to \SI{350}{\gev}.


\end{document}