\chapter{Fundamentals of Mono-Photon analysis}
Large missing transverse momentum (\met) signature can be sign of evidence of new physics, such as the indirect detection of dark matter.

At LHC we can produce DM particles if they interact with Standard Model particles. A possible candidate to be a constituent of dark matter in universe is given by a weakly interactive mass particle (WIMP) which interacts with SM particles with a strenght similar to the weak interaction so it leaves no track on the detector. Missing transverse momentum can only be measured  if other particles are produced in the collision, for instance a detectable object such jets or photons, in order to tag the WIMP particle production. This kind of analysis are called Mono-X, for one selects events with a single object in final state. In the Mono-Photon analysis we are looking for a single high energy photon and large missing transverse momentum signature. The monophoton analysis is characterized by a relative clean final state thanks also to a small set of SM processes that produces the same outcome.

For the analysis we are using data collected during 2015 and 2016 with the ATLAS detector, corresponding to an integrated luminosity of $36.4$ \ifb.

In this chapter after giving a definition of signal and control regions describing the kinematic cuts used to constrain data I present the result for a background only analysis. I give also a brief evaluation of the raise of systematic uncertainties.

\section{Event selection}
In the context of this MonoPhoton analysis several Conotrol Regions (CRs) enriched with the corresponding dominant background in order to extract their normalization on data and a Signal Region (SR) in which eventual signal could be found.

Events in SR are preselected required by
\begin{itemize}
\item belonging to the Good Run List (GRL) for data quality;
\item passing the  \verb!HLT_g140_loose! trigger
\item having a primary vertex reconstructed with two associated good-quality tracks with \pt $\ge 400 \,$\MeV and \AetaRange{2.5}
\item rejecting events with {\itshape BadLoose} jet (la [38] della nota) which overlaps with photons or leptons with \pt$> 20$ \GeV or not coming from hard scattering
\end{itemize}

After the preselection
Candidates in SR are selected with the following kinematic cuts. We consider an event to be part of this region if:
\begin{itemize}
\item we compute \met $ \ge 150 $ \GeV;
\item we get one loose photon with \pt $ \ge 150 $ \GeV $\,$ with a pseudorapidity cut in \etaRange{1.52}{2.37} to cut ot the calorimeter crack region and \AetaRange{1.37}
\item \met significance is found to be above $8.5$ \GeV$^{-\frac{1}{2}}$
\item the leading photon is isolated, so that we select events characterized by $ \text{TopoEtcone40} \le 2.45$ \GeV$ + 0.022 \, $\pt \GeV. By this we ensure that in a cone with radius \DeltaRdef $ = 40$, centered in the photon track, all the Topo Clusters have less energy deposited than the one defined above.
\item photon track and \met doesn't overlap, requiring $\Delta\phi_{\gamma - \met} \le 0.4$
\item photon pointing along z coordinate wrt the identified primary vertex must no be larger than $250 \, mm$
\item there is at most one {\itshape good} jet. Even if the analysis is called monophoton we take into account events even if they contains a jet, if not we could reject too much statistics as it will be clear in validation plots in section {\bfseries which hasn't been written yet}. Moreover we take only jets such that $\deltaphijetgamma \ge 0.4$

\end{itemize}

  

  
