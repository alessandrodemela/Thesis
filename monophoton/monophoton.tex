\chapter{Fundamentals of the \mph analysis}
Large missing transverse momentum (\met) signature can be sign of evidence of new physics, such as the indirect detection of dark matter.

At LHC we can produce DM particles if they interact with Standard Model particles. A possible candidate to be a constituent of dark matter in universe is given by a weakly interactive mass particle (WIMP) which interacts with SM particles with a strenght similar to the weak interaction so it leaves no track on the detector. Missing transverse momentum can only be measured  if other particles are produced in the collision, for instance a detectable object such jets or photons, in order to tag the WIMP particle production. This kind of analysis are called Mono-X, for one selects events with a single object in final state. In the \mph analysis we are looking for a single high energy photon and large missing transverse momentum signature. The \mph analysis is characterized by a relative clean final state thanks also to a small set of SM processes that produces the same outcome.

For the analysis we are using data collected during 2015 and 2016 with the ATLAS detector, corresponding to an integrated luminosity of $36.4$ \ifb.

In this chapter after giving a definition of signal and control regions describing the kinematic cuts used to constrain data I present the result for a background only analysis. I give also a brief evaluation of the raise of systematic uncertainties.

\section{Event selection}
In the context of this \mph analysis several Conotrol Regions (CRs) enriched with the corresponding dominant background in order to extract their normalization on data and a Signal Region (SR) in which eventual signal could be found.

\subsection{Selection in Signal Region}
Events in SR are preselected required by the following property:
\begin{description}%MAGARI CAMBIALO CON UNA DESCRIPTION
\item [Data quality] Data are considered good for physics if belonging to the Good Run List (GRL). They have to be collected in periods in which optimal detector functioning and stable beams were provided;
\item [Trigger acceptance] Events selected must pass the  \verb!HLT_g140_loose! trigger, which {\itshape I think} is a trigger used to detect events with large \met; 
\item [Vertex quality] We take into account events having a primary vertex reconstructed with two associated good-quality tracks with \pt $\ge 400 \,$\MeV and \AetaRange{2.5}
\item [Jet cleaning] Jets tagged with {\itshape BadLoose} (la [38] della nota) in events in which they overlap with photons or leptons with \pt$> 20$ \GeV or not coming from hard scattering are rejected.
\end{decription}

After the preselection Candidates in SR are selected with the following kinematic cuts. In order to populate the region with \gmet events, we consider an event to be part of this region if:
\begin{itemize}
\item we compute \met $ \ge 150 $ \GeV;
\item we get one loose photon with \pt $ \ge 150 $ \GeV $\,$ with a pseudorapidity cut in \etaRange{1.52}{2.37} to cut ot the calorimeter crack region and \AetaRange{1.37}
\item \met significance is found to be above $8.5$ \GeV$^{-\frac{1}{2}}$
\item the leading photon is isolated, so that we select events characterized by $ \text{TopoEtcone40} \le 2.45$ \GeV$ + 0.022 \, $\pt \GeV. By this we ensure that in a cone with radius \DeltaRdef $ = 40$, centered in the photon track, all the Topo Clusters have less energy deposited than the one defined above.
\item photon track and \met doesn't overlap, requiring $\Delta\phi_{\gamma - \met} \le 0.4$
\item photon pointing along z coordinate wrt the identified primary vertex must no be larger than $250 \, mm$
\item there is at most one {\itshape good} jet. Even if the analysis is called \mph we take into account events even if they contains a jet, if not we could reject too much statistics as it will be clear in validation plots in section {\bfseries which hasn't been written yet}. Moreover we take only jets such that $\deltaphijetgamma \ge 0.4$

\end{itemize}

In the following section I'm going to deal with all the sources of interference that could affect the SR describing the background processes.

\section{Background estimation}
Alongside the eventual signal coming from unknown ph\oe nomena many other SM processes could pass the cuts defined for the SR ending up being tagged in a wrong way as new physics. We call them background signal, i.e. predictable events that lead to the same signature we are looking for when producing DM particles from \pp scattering.

In other words reasons for events of background can be ascribed to:
\begin{itemize}
\item properly reconstructed SM events which have the same final state as the signal we are looking for;
\item badly reconstructed events in which one (or more) particle are mistaken for photons or they are not reconstructed at all.
\end{itemize}

For the \mph analysis the following background processes were considered.
\begin{itemize}
\item \znng where the two neutrinos produce high \met in the detector. This is the dominant, other than the only irreducible, background as one can see from plots.
\item \wg in which all possible leptonic decay mode of \Wboson are gathered. Here $\ell$ can be an electron being missed or reconstructed as a photon, a muon being missed as well, or a $\tau$ lepton which can decay via hadrons and reconstructed as jet or via leptons and being missed in the detector.
\item \zg where, once again, I am considering all possible leptonic decays for the \Zboson where leptons end up in the same records as in the \Wboson case
\item \gj events in which a jet or photon misreconstruction leads to fake \met.
\item All kind of processes where a jet can fake a photon including:
  \begin{itemize}
  \item double neutrino decay of \Zboson combined with a jet,
  \item leptonic decay of \Wboson along with a jet. If an electron is involved it could fake a photon as well, while we are keeping the jet as pure object for passing SR cuts.
  \end{itemize}
\end{itemize}

In order to quantify the amount of background events in SR, several Control Regions (CRs) are defined which are assumed to be free of signal contamination. Each of them is defined reverting one or more constraints for the SR to enrich every region with a given background process so that they are characterized by a pure dominant process among the others. This provides also ortogonality of a region to one another and their statistical independence.

Events in CRs are simulated via Monte Carlo samples whose prediction must be fitted on data, to get a more reliable estimate of background in SR. This procedure will be explained in detail in section {\bfseries to be written}. A special procedure must be carried out for electron and jets faking photons. In this work results in all the regions for these sources have been taken from previous analysis. However this kind of background cannot be quantified like the previous ones because it is due to some detector mistakes in reconstruction.



\subsection{Transfer factor technique and normalisation of background}
In CRs the respective dominant background can be controlled by comparing MC samples to data. Initial predictions of background in a certain CR must be scaled to observed data in that region, using a normalization factor ($k_{i}$) which is usually computed by a numerical fit by the HistFitter software framework. Usually this procedure is referred as an extrapolation from CRs to SR.

One can define a Trasfer Factor by a mere ratio between MC prediction in SR and MC prediction in a CRs, both unnormalized (raw), and the estimate of the number of events in the signal region for the $i$-th process could be written in terms of event observed in the CRs as:
\begin{equation}
  \label{eqn:TF}
  N_{\textup{est},i}(\text{SR}) =  N_{\textup{obs},i}({\text{CR}}) \times \left[\frac{\text{MC}_{\textup{raw},i}(\text{SR})}{\text{MC}_{\textup{raw},i}(\text{CR})} \right]
\end{equation}

The proper normalization factor is defined as the ratio between events observed and expected in the CRs and equation \ref{eqn:TF} can also be rewritten:
\begin{equation}
  N_{\textup{est},i}(\text{SR}) = k_{i}\times\text{MC}_{\textup{raw},i}(\text{SR})
\end{equation}

By this factor the PDF of every region is rescaled everywhere in the parameter space. Indeed the analysis strategy performed by HistFitter shares the same parameters in all region so that any information found in a single region can be brought to the others.

The advantage of using this procedure is that that systematic uncertainties common to the numerator and the denominator of the TF, such as the uncertainty on luminosity, on the predicted background processes can be canceled at first order in the extrapolation.

\subsection{Definition of Control Regions}
There are four CRs definded in order to extrapolate information on background in Signal Region. In this section a brief definition and differences in kinematic cuts are descrbed.

\begin{description}
\item [One muon CR] In this region one selected muon is requested, kinematic cuts on \pt and \met are the same of those in SR. Here, however, \met is computed in a different way that is the muon energy is added to this variable, so we can treat it as an invisible particle in order to ensure that \met distribution is similar to the one in SR. This region extracts the normalization $k_W$ for \wg background.
\item [Two muon/electron CR] Similarly to the One muon, here we add to \met both muon/electron contribution to \met. Here two selected muons/electrons are required. we can also add a constrain to the invariant mass of the di-lepton to be greater than 20 \GeV\, and less than 1 \TeV \, to avoid any BSM $Z\gamma$ resonances. Moreover we don't require the cut on \met significance. These regions get $k_Z$ for the dominant background \zg in this region which applies via branching ratio to the \znng process.
\item [Photon jet CR] Here kinematic cuts on \met are lowered. We require \met to be in 85 - 110 \GeV\, range to enrich this region of \gj background because in this energy range, probability to have fake \met from jets is higher. To avoid signal contamination we set $\Delta\phi_{\gamma - \met} \le 3.0$ so that we prevent {\itshape back-to-back} signal events to fall in this region. This region, defined in \RunTwo, constrains the normalization for \gj background to get a $k_{pj}$ normalization factor.
\end{description}

{\itshape A priori} a one electron CR could have been defined as well. Nevertheless the one muon CR has enough statistic to constrain the normalization on \Wboson$\gamma$.  This kind of region would have been more difficult to manage even just because electrons could be reconstructed easily as photons which not happens for muons being detected far away in the detector from photons increasing the number of \gj events contamination. In the analysis carried out in [nota di supporto] one evince that no improvement on $k_W$ error can be seen, instead there is a raise of $k_{pj}$ error due to furhter background contamination.

As mentioned earlier fake photons from electrons and jets enter the fit as given values and it is not constrained by any region. They have been estimated via data driven tecniques in other analysis.

\section{Results for a background-only fit}
{\bfseries To be written}
