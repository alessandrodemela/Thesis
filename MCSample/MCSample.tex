\chapter[Monte Carlo simulation]{Monte Carlo simulation samples for the Minimal Dark Matter model}

The process of sample generation comes across several steps. The first generation of sample in which the very fisrt products of \pp collision are simulated followed by the \emph{hadronization} and the \emph{parton-shower}. Up to now the Truth-level simulation is accomplished and a readable file (a TRUTH file) can be built from this events. 

On the other hand a complete simulation can be pursued. For this purpose, after the hadronization took place, the simulation of the detector comes afterwards followed by a first reconstruction of events . Then all the events reconstructed are filtered by a derivation code which selects only the  relevant events for this analysis. Finally the ``derivated'' events are collected in a compact and practical RooT file called \emph{mini-tree}.

\section{Non so se descrivere qui il minimal model}

\section{Generation}
The Minimal model has alreasy been implemented in an UFO file containing its details [cit. al talk di Marta Perego] considering the pure electroweak triplet ($\chi^+$,$\chi_0$, $\chi^-$) and implemented in FeynRules. 

\section{TRUTH1 file}
\lipsum

\section{Simulation}
\lipsum

\section{Reconstruction}
\lipsum

\section{EXOT6 Derivation}
\lipsum

\section{Mini-trees}
\lipsum