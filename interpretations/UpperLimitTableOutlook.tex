
\begin{table}[pt]
\centering
%\setlength{\tabcolsep}{0.0pc}
\begin{tabular}{ccccc}
\noalign{\smallskip}\toprule\noalign{\smallskip}
L [\ifb]&$\left({\rm \sigma_{\rm vis}}\right)_{\rm obs}^{95}$~[fb] &$\left({\rm \sigma_{\rm vis}}\right)_{\rm exp}^{95}$~[fb] &  $N_{\rm obs}^{95}$  & $N_{\rm exp}^{95}$  \\
\noalign{\smallskip}\midrule\noalign{\smallskip}
%%
$120$&$1.79$ &${ 2.69 }^{ +1.00 }_{ -0.73 } $ &  $215.1$ & $ { 322.4 }^{ +120.5 }_{ -87.0 }$ \\
\noalign{\smallskip}\noalign{\smallskip}
$3000$&$0.072$ &${ 0.108 }^{ +0.040 }_{ -0.029 } $ &  $215.2$ & $ { 322.8 }^{ +120.5 }_{ -87.0 }$ \\

\noalign{\smallskip}\bottomrule\noalign{\smallskip}
\end{tabular}
\caption[Breakdown of upper limits.]{
Outlook at future luminosity of upper limits on the cross section and number of events.\\
Left to right: First column reports the integrated luminosity used, then \SI{95}{\percent} CL upper limits on the visible cross section observed
followed by the expected
reporting the $\pm 1\sigma$
excursions are given. The third column shows the \SI{95}{\percent} CL upper limit on on the number of observed
signal events while the last reports the \SI{95}{\percent} CL upper limit on the expected number with $\pm 1\sigma$
excursions on the expectation, of background events.
Expected limits are given the absence of new physics.
\label{table.results.exclxsec.pval.upperlimit.SR.outlook}}
\end{table}
%