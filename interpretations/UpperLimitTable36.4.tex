
\begin{table}
\centering
%\setlength{\tabcolsep}{0.0pc}
\begin{tabular}{cccc}
\noalign{\smallskip}\toprule\noalign{\smallskip}
$\left({\rm \sigma_{\rm vis}}\right)_{\rm obs}^{95}$~[fb] &$\left({\rm \sigma_{\rm vis}}\right)_{\rm exp}^{95}$~[fb] &  $N_{\rm obs}^{95}$  & $N_{\rm exp}^{95}$  \\
\noalign{\smallskip}\midrule\noalign{\smallskip}
%%
$5.908$ &${ 8.863 }^{ +3.310 }_{ -2.390 } $ &  $215.1$ & $ { 322.6 }^{ +120.5 }_{ -87.0 }$ \\
 
 
\noalign{\smallskip}\bottomrule\noalign{\smallskip}
\end{tabular}
\caption[Breakdown of upper limits.]{
Left to right:  \SI{95}{\percent} CL upper limits on the visible cross section observed
followed by the expected
reporting the $\pm 1\sigma$
excursions. The third column shows the \SI{95}{\percent} CL upper limit on on the number of observed
signal events while the last reports the \SI{95}{\percent} CL upper limit on the expected number with $\pm 1\sigma$
excursions on the expectation, of background events.
Expected limits are given the absence of new physics.
\label{table.results.exclxsec.pval.upperlimit.SR}}
\end{table}
%