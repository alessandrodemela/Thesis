
\begin{table}[pt]
\centering
%\setlength{\tabcolsep}{0.0pc}
\begin{tabular}{ccc}
\noalign{\smallskip}\toprule\noalign{\smallskip}
L [\ifb]&$\rm \sigma_{\rm vis}^{\rm obs}$~[fb] &$\rm \sigma_{\rm vis}^{\rm exp}$~[fb] \\
\noalign{\smallskip}\midrule\noalign{\smallskip}
%%
$36.4$&$5.91$ &${ 8.86 }^{ +3.31 }_{ -2.39 } $ \\
\noalign{\smallskip}\midrule[.2pt]\noalign{\smallskip}
$120$&$-$ &${ 2.69 }^{ +1.00 }_{ -0.73 } $ \\
\noalign{\smallskip}\noalign{\smallskip}
$3000$&$-$ &${ 0.108 }^{ +0.040 }_{ -0.029 } $ \\
 
 
\noalign{\smallskip}\bottomrule\noalign{\smallskip}
\end{tabular}
\caption[Breakdown of upper limits.]{
Left to right:  First column reports the integrated luminosity used, then \SI{95}{\percent} CL upper limits on the visible cross section observed
followed by the expected
reporting the $\pm 1\sigma$
excursions are given. 
Expected limits are given the absence of new physics.
First row of the table reports the result achieved in this work and found to be in good agreement with thise published in~\cite{paperMP}, and below the solid line
outlooks at high luminosity is given.
\label{table.results.exclxsec.pval.upperlimit.SR}}
\end{table}
%